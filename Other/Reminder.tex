%\documentclass[a4paper,10pt,oneside]{article}
%\usepackage{xeCJK}
%\setCJKmainfont{PingFangTC-Regular}
%\begin{document}

\subsubsection{Complexity}

\begin{enumerate}\itemsep = -5pt

\item LCA
\begin{tabbing}
Method........ \= Time..................... \= Space............ \=  離線\\
SsadpTarjan  \> $O(N + Q)$          \> $O(N ^ 2)$      \>  不須離線\\
OfflineTarjan \> $O(N + Q)$           \> $O(N + Q)$    \>  須離線  \\
SparseTable \> $O(N + Q \log N)$ \> $O(N \log N)$ \>不須離線
\end{tabbing}

\item Dinic
\begin{tabbing}
Graph.......... \= Space........ \= Time\\
Gernal          \> $O(V+E)$ \> $O(EV^2)$  \\
Bipartite        \> $O(V+E)$ \> $O(E\sqrt V)$ \\
UnitNetwork  \> $O(V+E)$ \> $O(E \min(V^{1.5}, \sqrt E))$
\end{tabbing}

\end{enumerate}

\subsubsection{二分圖匹配}

\begin{enumerate}\itemsep = -5pt
\item 最大匹配數:給定二分圖 $G$,在 $G$ 的子圖 $M$ 中,$M$ 的任兩條邊都沒有公共節點,則 $M$ 成為此二分圖的匹配,$|EM|$ 最大的匹配則成為最大匹配。
\item 最小點覆蓋:在 $VG$ 中選取最少的點,形成子集合 $V$,使 $E$ 為所有與 $V$ 中的點 incident 的邊形成的集合。
\item 最大獨立集:在 $VG$ 中選取最多的點,形成子集合 $V$,且任兩個 $V$ 中的 vertices 都不相鄰。
\item Konig 定理:對於任意二分圖,滿足以下兩個條件
\begin{enumerate} \itemsep = -3pt
\item 最大匹配數 = 最小點覆蓋的頂點數
\item 最大獨立集之頂點數 = 總頂點數 - 最大匹配數
\end{enumerate}
\end{enumerate}

\subsubsection{Pick公式}
給定頂點坐標均是整點的簡單多邊形,面積=內部格點數+邊上格點數/2-1

\subsubsection{圖論}

\begin{enumerate}\itemsep = -5pt
\item 中國郵差問題
	\begin{enumerate} \itemsep = -3pt
	\item 先判斷整張圖是否為一個強連通分量,否則無解。
	\item 找出圖上所有奇點,一定是偶數個。
	\item 找出所有奇點點對之間的最短路徑長度。
	\item 把這些奇點做最小權匹配,權重採用剛才算的最短路徑長度。
	\item 把匹配邊加在原圖上,再找歐拉環,即得中國郵差路徑之權重。
	\item 將匹配邊改成其代表的最短路徑,即得中國郵差路徑。
	\end{enumerate}
\item For planner graph,$F=E-V+C+1$,C是連通分量數
\item For planner graph,$E\leq 3V-6$
\item 對於連通圖G,最大獨立點集的大小設為I(G),最大匹配大小設為M(G),最小點覆蓋設為Cv(G),最小邊覆蓋設為Ce(G)。
對於任意連通圖:
	\begin{enumerate}\itemsep = -3pt
	\item $I(G)+Cv(G)=|V|$
	\item $M(G)+Ce(G)=|V|$
	\end{enumerate}
\item 對於連通二分圖:
	\begin{enumerate}\itemsep = -3pt
	\item $I(G)=Cv(G)$
	\item $M(G)=Ce(G)$
	\end{enumerate}
\item 最大權閉合圖:
	\begin{enumerate}\itemsep = -3pt
	\item $C(u,v)=\infty ,(u,v)\in E$
	\item $C(S,v)=W_v ,W_v>0$
	\item $C(v,T)=-W_v ,W_v<0$
	\item ans=$\sum_{W_v>0} W_v-flow(S,T)$
	\end{enumerate}
\item 最大密度子圖:
	\begin{enumerate}\itemsep = -1pt
	\item 求$max\left(\frac{W_e+W_v}{|V'|}\right),e \in E',v \in V'$
	\item $U=\sum_{v \in V}2W_v+\sum_{e \in E} W_e$
	\item $C(u,v)=W_{(u,v)} ,(u,v)\in E$,雙向邊
	\item $C(S,v)=U ,v \in V$
	\item $D_u=\sum_{(u,v) \in E} W_{(u,v)}$
	\item $C(v,T)=U+2g-D_v-2W_v ,v \in V$
	\item 二分搜$g$:\\$l=0,r=U,eps=1/n^2$\\if($(U\times|V|-flow(S,T))/2>0$) $l=mid$\\else $r=mid$
	\item ans=$min\_cut(S,T)$
	\item $|E|=0$要特殊判斷
	\end{enumerate}
\item 弦圖:
	\begin{enumerate}\itemsep = -3pt
	\item 點數大於3的環都要有一條弦
	\item 完美消除序列從後往前依次給每個點染色,給每個點染上可以染的最小顏色
	\item 最大團大小=色數
	\item 最大獨立集:完美消除序列從前往後能選就選
	\item 最小團覆蓋:最大獨立集的點和他延伸的邊構成
	\item 區間圖是弦圖
	\item 區間圖的完美消除序列:將區間按造又端點由小到大排序
	\item 區間圖染色:用線段樹做
	\end{enumerate}
\end{enumerate}

\subsubsection{0-1分數規劃}
$x_i=\{0,1\}$,$x_i$可能會有其他限制,求$max\left(\frac{\sum B_ix_i}{\sum C_ix_i}\right)$
\begin{enumerate}\itemsep = -1pt
\item $D(i,g)=B_i-g\times C_i$
\item $f(g)=\sum D(i,g)x_i$
\item $f(g)=0$時$g$為最佳解,$f(g)<0$沒有意義
\item 因為$f(g)$單調可以二分搜$g$
\item 或用Dinkelbach通常比較快
\end{enumerate}
\begin{lstlisting}[language=C++]
binary_search(){
	while(r-l>eps){
		g=(l+r)/2;
		for(i:所有元素)D[i]=B[i]-g*C[i];//D(i,g) 
		找出一組合法x[i]使f(g)最大;
		if(f(g)>0) l=g;
		else r=g;
	}
	Ans = r;
}
Dinkelbach(){
	g=任意狀態(通常設為0);
	do{
		Ans=g;
		for(i:所有元素)D[i]=B[i]-g*C[i];//D(i,g) 
		找出一組合法x[i]使f(g)最大;
		p=0,q=0;
		for(i:所有元素)
			if(x[i])p+=B[i],q+=C[i];
		g=p/q;//更新解,注意q=0的情況 
	}while(abs(Ans-g)>EPS);
	return Ans;
}
\end{lstlisting}

\subsubsection{Math}
\begin{enumerate}\itemsep = -3pt
\item $\sum_{d|n} \phi(n) = n$
\item $\text{Harmonic series } H_n = \ln(n) + \gamma + 1/(2n) - 1/(12n^2) + 1/(120n^4)$
\item Gray Code $=n\oplus (n>>1)$
\item $SG(A+B)=SG(A)\oplus SG(B)$
\item Rotate Matrix $M(\theta)= \left( \begin{array}{ccc}
cos\theta & -sin\theta \\ 
sin\theta &  cos\theta
\end{array} \right)$
\item $\sum_{d|n} \mu (n)=[n==1]$
\item $g(m)=\sum_{d|m}f(d)\Leftrightarrow f(m)=\sum_{d|m}\mu (d) \times g(m/d)$
\item $\sum_{i=1}^n\sum_{j=1}^m$互質數量$=\sum \mu (d)\left \lfloor \frac{n}{d} \right \rfloor \left \lfloor \frac{m}{d} \right \rfloor$
\item $\sum_{i=1}^n\sum_{j=1}^nlcm(i,j)=n\sum_{d|n} d \times \phi (d)$
\item Josephus Problem\\
$f(1,k) = 0, f(n, k) = (f(n-1,k)+k) \% n$
\item Mobius\\
$u(n) = \begin{cases}
1 &, n = 1\\
0 &, n \text{有平方數因數}\\
(-1)^k &, n = p_1 p_2 p_3 \dots p_k
\end{cases}$\\
$u(ab) = u(a)u(b), \sum_{d | n} u(d) = [n == 1]$
\item Mobius Inversion\\
$f(m)=\sum_{d|n}g(d)\Leftrightarrow g(n)=\sum_{d|n} u(d) \times f(n/d) = \sum_{d|n} u(n/d) \times f(d) $

\item {排組公式}
\begin{enumerate}
\itemsep = -3pt
\item n-Catalan $C_0 = 1$,$C_{n+1} = \frac{2(2n+1)C_n}{n+2}$
\item kn-Catalan $\frac{C_{n}^{kn}}{n(k-1)+1}$,$C_{m}^{n}=\frac{n!}{m!(n-m)!}$
\item Stirling number of $2^{nd}$,$n$人分$k$組方法數目
	\begin{enumerate}\itemsep = -2pt
		\item $S(0,0)=S(n,n)=1$
		\item $S(n,0)=0$
		\item $S(n,k)=kS(n-1,k)+S(n-1,k-1)$
	\end{enumerate}
\item Bell number,$n$人分任意多組方法數目
	\begin{enumerate}\itemsep = -2pt
		\item $B_0=1$
		\item $B_n=\sum_{i=0}^nS(n,i)$
		\item $B_{n+1}=\sum_{k=0}^{n} C_k^n B_k$
		\item $B_{p+n}\equiv B{_n}+B_{n+1} mod p$, p is prime
		\item $B_{p^m+n}\equiv mB{_n}+B_{n+1} mod p$, p is prime
		\item From $B_0: 1,1,2,5,15,52,\\203,877,4140,21147,115975$
	\end{enumerate}
\item Derangement,錯排,沒有人在自己位置上
	\begin{enumerate}\itemsep = -2pt
		\item $D_n=n!(1-\frac{1}{1!}+\frac{1}{2!}-\frac{1}{3!}\ldots +(-1)^n\frac{1}{n!})$
		\item $D_n=(n-1)(D_{n-1}+D_{n-2}),D_0=1,D_1=0$
		\item From $D_0: 1,0,1,2,9,44,\\265,1854,14833,133496$
	\end{enumerate}
\item Binomial\ Equality
	\begin{enumerate}\itemsep = -2pt
	    \item $\sum_k \binom{r}{m + k} \binom{s}{n - k} = \binom{r + s}{m + n}$
         \item $\sum_k \binom{l}{m + k} \binom{s}{n + k} = \binom{l + s}{l -m + n}$		
         \item $\sum_k \binom{l}{m + k} \binom{s + k}{n}(-1)^k = (-1)^{l + m} \binom{s - m}{n - l}$
		\item $\sum_{k\leq l} \binom{l - k}{m} \binom{s}{k - n}(-1)^k = (-1)^{l + m} \binom{s - m - 1}{l - n - m}$
		\item $\sum_{0 \leq k \leq l} \binom{l - k}{m} \binom{q + k}{n} = \binom{l + q + 1}{m + n + 1}$
		\item $\binom{r}{k} = (-1)^k\binom{k - r - 1}{k}$
		\item $\binom{r}{m} \binom{m}{k} = \binom{r}{k} \binom{r - k}{m - k}$
		\item $\sum_{k\leq n} \binom{r + k}{k} = \binom{r + n + 1}{n}$
		\item $\sum_{0\leq k \leq n} \binom{k}{m} = \binom{n + 1}{m + 1}$
		\item $\sum_{k\leq m}\binom{m + r}{k}x^ky^k = \sum_{k\leq m}\binom{-r}{k}(-x)^k (x+y)^{m-k}$	
	\end{enumerate}
\end{enumerate}
	
\item {LinearAlgebra}
\begin{enumerate} \itemsep = -3pt
\item $tr(A) = \sum_{i} A_{i, i}$
\item eigen vector: $(A - cI) x = 0$ 
\end{enumerate}

\item{冪次,冪次和}
\begin{enumerate}\itemsep = -3pt
	\item $a^b\%P=a^{b\% \varphi (p)+\varphi (p)},b\geq \varphi (p)$
	\item $1^3+2^3+3^3+\ldots +n^3=\frac{n^4}{4}+\frac{n^3}{2}+\frac{n^2}{4}$
	\item $1^4+2^4+3^4+\ldots +n^4=\frac{n^5}{5}+\frac{n^4}{2}+\frac{n^3}{3}-\frac{n}{30}$
	\item $1^5+2^5+3^5+\ldots +n^5=\frac{n^6}{6}+\frac{n^5}{2}+\frac{5n^4}{12}-\frac{n^2}{12}$
	\item $0^k+1^k+2^k+\ldots +n^k = P(k),P(k)=\frac{(n+1)^{k+1}-\sum_{i=0}^{k-1}C_i^{k+1}P(i)}{k+1},P(0)=n+1$
	\item $\sum_{k=0}^{m-1}k^n=\frac{1}{n+1}\sum_{k=0}^{n}C_k^{n+1}B_km^{n+1-k}$
	\item $\sum_{j=0}^{m}C_j^{m+1}B_j=0,B_0=1$
	\item 除了$B_1=-1/2$,剩下的奇數項都是$0$
	\item $B_2=1/6,B_4=-1/30,B_6=1/42,B_8=-1/30,B_{10}=5/66,B_{12}=-691/2730,B_{14}=7/6,B_{16}=-3617/510,B_{18}=43867/798,B_{20}=-174611/330,$
\end{enumerate}

\item{Chinese Remainder Theorem}
\begin{enumerate}\itemsep = -3pt
\item $gcd(m_i, m_j) = 1$
\item
$x \% m_1 = a_1$\\
$x \% m_2 = a_2$\\
$\vdots$ \\
$x \% m_n = a_n$
\item $M = m_1 m_2 \dots m_n, M_i = M / m_i$
\item $t_i m_i = 1 (\mod m_i)$
\item $x = a_1 t_1 * M_1 + \dots + a_n t_n * M_n + k M, k \in N$
\end{enumerate}

\end{enumerate}


\subsubsection{Burnside's lemma}
\begin{enumerate}\itemsep = -3pt
	\item $|X/G| = \frac{1}{|G|}\sum_{g \in G}|X^g|$
	\item $X^g=t^{c(g)}$
	\item $G$表示有幾種轉法,$X^g$表示在那種轉法下,有幾種是會保持對稱的,$t$是顏色數,$c(g)$是循環節不動的面數。
	\item 正立方體塗三顏色,轉0有$3^6$ 個元素不變,轉90有6種,每種有$3^3$不變,180有$3\times 3^4$,120(角)有$8\times 3^2$,180(邊)有$6\times 3^3$,全部$\frac{1}{24}\left(3^6+6\times 3^3 + 3 \times 3^4 + 8 \times 3^2 + 6 \times 3^3 \right) = 57$
\end{enumerate}

\subsubsection{Probability}
\begin{enumerate}\itemsep = -3pt
	\item $e^x (1 - x^2) \leq 1 + x \leq e^x$
	\item $n! \leq en^{\frac{1}{2}} (\frac{n}{e})^n$
	\item $Pr[X \geq a] \leq \displaystyle\frac{E[X]}{a}, X \le 0, a > 0$
	\item $\text{Cov}[X, Y] = E[(X - E[X])(Y - E[Y])] = E[XY] - E[X]E[Y]$
	\item $\displaystyle \text{Var}[\sum_j X_j] = (\sum_j\text{Var}[X_j]) + 2\sum_{i < j}  \text{Cov}[X_i, X_j]$
	\item $Pr[X \leq a] \leq \displaystyle \min_{t < 0} \frac{E[e^{tX}]}{e^{ta}}$
	\item $M_X(t) = E[e^{tX}]$
	\item $Pr[X \geq a] \leq \displaystyle \min_{t > 0} \frac{E[e^{tX}]}{e^{ta}}$
	\item $Pr[X \leq a] \leq \displaystyle \min_{t < 0} \frac{E[e^{tX}]}{e^{ta}}$
	\item $\forall \delta > 0, Pr[X \geq (1 + \delta) \mu] \leq \displaystyle(\frac{e^{\delta}}{(1 + \delta)^{(1 + \delta)}})^\mu$
	\item $\forall 0 < \delta \leq 1, Pr[X \geq (1 + \delta) \mu] \leq  e^{\frac{-\mu\delta^2}{3}}$
	\item $R \geq 6\mu, Pr[X \geq R] \leq 2^{-R}$
	\item $0 < \delta < 1, Pr[X \leq (1 - \delta) \mu] \leq  (\frac{e^{-\delta}}{(1 - \delta)^{(1 - \delta)}})^{\mu}$
	\item $0 < \delta \le 1, Pr[X \leq (1 - \delta) \mu] \leq  e^{\frac{-\mu\delta^2}{2}}$
\end{enumerate}

\subsubsection{Tree Counting}
\begin{enumerate}\itemsep = -3pt
	\item Rooted tree: $s_{n+1}=\frac{1}{n}\sum_{i=1}^{n}(i\times a_i\times \sum_{j=1}^{\left \lfloor  n/i\right \rfloor} a_{n+1-i\times j})$
	\item Unrooted tree: 
	\begin{enumerate}\itemsep = -2pt
		\item Odd:$a_n-\sum_{i=1}^{n/2}a_ia_{n-i}$
		\item Even:$Odd+\frac{1}{2}a_{n/2}(a_{n/2}+1)$
	\end{enumerate}
	\item Spanning Tree
	\begin{enumerate}\itemsep = -2pt
		\item Cayley: $n^{n-2}$ (Complete Graph)
		\item Kirchhoff: $M[i][i]=\deg(V_i)$,$M[i][j]= E(i,j) ? -1 : 0$. delete any one row and col in $A$, $ans = det(A)$
	\end{enumerate}
\end{enumerate}

%\end{document}